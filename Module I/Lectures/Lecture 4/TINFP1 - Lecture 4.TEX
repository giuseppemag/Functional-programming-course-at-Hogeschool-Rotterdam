\documentclass{beamer}
\usetheme{Warsaw}
\usepackage{hrslides}
\usepackage{graphicx}
\usepackage[space]{grffile}
\usepackage{listings}
\lstset{language=C,
basicstyle=\ttfamily\footnotesize,
frame=shadowbox,
mathescape=true,
showstringspaces=false,
showspaces=false,
breaklines=true}
\usepackage[utf8]{inputenc}

\title{Trees and expressions}

\author{Dr. Giuseppe Maggiore}

\institute{Hogeschool Rotterdam \\ 
Rotterdam, Netherlands}

\date{}



\begin{document}
\maketitle

\SlideSection{Introduction}
\SlideSubSection{Lecture topics}
\begin{slide}{
\item Recursive definition of binary search trees
\item Recursive traversal of trees with pattern matching
\item Recursive definition of expressions
\item Building an interpreter
}\end{slide}

\SlideSubSection{Definition and basic traversals}
\begin{frame}[fragile]
\begin{lstlisting}
type BinTree<'T> =
  | Leaf
  | Node of 'T * BinTree<'T> * BinTree<'T>
\end{lstlisting}
\end{frame}

\begin{frame}[fragile]
\begin{lstlisting}
let rec add x t =
  match t with
  | Leaf -> Node(x, Leaf, Leaf)
  | Node(y,l,r) when x < y ->
    Node(y,add x l,r)
  | Node(y,l,r) when x > y ->
    Node(y,l,add x r)
  | _ -> t
\end{lstlisting}
\end{frame}

\begin{frame}[fragile]
\begin{lstlisting}
let t = Leaf |> add 3 |> add 5 |> add 10 |> add 7 |> add 1 |> add 2
\end{lstlisting}
\end{frame}

\begin{frame}[fragile]
\begin{lstlisting}
let (!) x = x, x.ToString()
let t = Leaf |> add !3 |> add !5 |> add !10 |> add !7 |> add !1 |> add !2
\end{lstlisting}
\end{frame}

\begin{frame}[fragile]
\begin{lstlisting}
let rec depth t =
  match t with
  | Leaf -> 0
  | Node(y,l,r) ->
    1 + max (depth l) (depth r)
\end{lstlisting}
\end{frame}

\begin{frame}[fragile]
\begin{lstlisting}
let rec minElem t =
  match t with
  | Leaf -> None
  | Node(y,Leaf,r) -> Some y
  | Node(y,l,r) -> minElem l
\end{lstlisting}
\end{frame}

\begin{frame}[fragile]
\begin{lstlisting}
let rec maxElem t =
  match t with
  | Leaf -> None
  | Node(y,l,Leaf) -> Some y
  | Node(y,l,r) -> maxElem r
\end{lstlisting}
\end{frame}

\begin{frame}[fragile]
\begin{lstlisting}
let rec find k t =
  match t with
  | Leaf -> None
  | Node((y,z),l,r) when k < y ->
    find k l
  | Node((y,z),l,r) when k > y ->
    find k r
  | Node((y,z),l,r) -> Some z
\end{lstlisting}
\end{frame}

\begin{frame}[fragile]
\begin{lstlisting}
let rec sum t =
  match t with
  | Leaf -> 0
  | Node(x,l,r) ->
    x + sum l + depth r
\end{lstlisting}
\end{frame}

\begin{frame}[fragile]
\begin{lstlisting}
let rec interval m M t =
  seq{
    match t with
    | Leaf -> ()
    | Node(x,l,r) when x < m ->
      yield! interval m M r
    | Node(x,l,r) when x > M ->
      yield! interval m M l
    | Node(x,l,r) ->
      yield! interval m M l
      yield x
      yield! interval m M r
  }
\end{lstlisting}
\end{frame}

\SlideSubSection{Representing expressions}
\begin{slide}{
\item A special kind of trees is used to represent syntactic expressions
\item By folding over the tree we can evaluate the expression tree
}\end{slide}

\begin{frame}[fragile]
\begin{lstlisting}
type Expr = 
  | Const of int
  | Sum of Expr * Expr
  | Mul of Expr * Expr
\end{lstlisting}
\end{frame}

\begin{frame}[fragile]
\begin{lstlisting}
let (!!) x = Const x
let (.+) e1 e2 = Sum(e1, e2)
let (.*) e1 e2 = Mul(e1, e2)
\end{lstlisting}
\end{frame}

\begin{frame}[fragile]
\begin{lstlisting}
let e = !!1 .+ (!!3 .* !!4)
\end{lstlisting}
\end{frame}

\begin{frame}[fragile]
\begin{lstlisting}
let rec eval e =
  match e with
  | Const c -> c
  | Sum(e1,e2) -> eval e1 + eval e2
  | Mul(e1,e2) -> eval e1 * eval e2
\end{lstlisting}
\end{frame}

\SlideSubSection{Conclusions and assignment}
\begin{slide}{
\item The assignments are on Natschool
\item \textbf{Restore} the games to a working state
\item Hand-in a \textbf{printed} report that only contains your \textbf{sources} and the associated \textbf{documentation}
}\end{slide}

\begin{slide}{
\item Any book on the topic will do
\item I did write my own (Friendly F\#) that I will be loosely following for the course, \textbf{but it is absolutely not mandatory or necessary to pass the course}
}\end{slide}


\begin{frame}{Dit is het}
\center
\fontsize{18pt}{7.2}\selectfont
The best of luck, and thanks for the attention!
\end{frame}

\end{document}
