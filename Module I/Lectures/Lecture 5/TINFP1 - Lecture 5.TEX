\documentclass{beamer}
\usetheme{Warsaw}
\usepackage{hrslides}
\usepackage{graphicx}
\usepackage[space]{grffile}
\usepackage{listings}
\lstset{language=C,
basicstyle=\ttfamily\footnotesize,
frame=shadowbox,
mathescape=true,
showstringspaces=false,
showspaces=false,
breaklines=true}
\usepackage[utf8]{inputenc}

\title{Records and abstract data types}

\author{Dr. Giuseppe Maggiore}

\institute{Hogeschool Rotterdam \\ 
Rotterdam, Netherlands}

\date{}

\begin{document}
\maketitle

\SlideSection{Introduction}
\SlideSubSection{Lecture topics}
\begin{slide}{
\item Abstract data types and interfaces as records of functions
\item Abstract operations as requiring records of functions to work
}\end{slide}

\SlideSubSection{Generalising over operations}
\begin{frame}[fragile]
\begin{lstlisting}
let add x y = x + y
\end{lstlisting}
\end{frame}

\begin{frame}[fragile]
\begin{lstlisting}
let a = add 1.0 2.0
\end{lstlisting}
\end{frame}

\begin{frame}[fragile]
\begin{lstlisting}
let b = add 1 2
let c = add 1.0f 2.0f
let d = add "one" "two"
\end{lstlisting}
\end{frame}

\begin{frame}[fragile]
\begin{lstlisting}
let add (+) x y = x + y
\end{lstlisting}
\end{frame}

\begin{frame}[fragile]
\begin{lstlisting}
let a = add (+) 1.0 2.0
let b = add (+) 1 2
let c = add (+) 1.0f 2.0f
let d = add (+) "one" "two"
\end{lstlisting}
\end{frame}

\begin{frame}[fragile]
\begin{lstlisting}
let line (+) (*) a b x =
  a * x + b
\end{lstlisting}
\end{frame}

\begin{frame}[fragile]
\begin{lstlisting}
let b = line (+) (*) 1.0 2.0 3.0
\end{lstlisting}
\end{frame}

\begin{frame}[fragile]
\begin{lstlisting}
let line (+) (*) a b x =
  a * x + b
\end{lstlisting}
\end{frame}

\begin{frame}[fragile]
\begin{lstlisting}
let a = line (+) (*) 1 2 3
let b = line (+) (*) 1.0 2.0 3.0
\end{lstlisting}
\end{frame}

\SlideSubSection{Grouping operations together}
\begin{slide}{
\item Instead of the single operations, we could give a record of them
\item This record is an abstraction over all the data types that will be able to support its operations
}\end{slide}

\begin{frame}[fragile]
\begin{lstlisting}
type NumberOperations<'a> = 
  { plus  : 'a -> 'a -> 'a
    times : 'a -> 'a -> 'a }
\end{lstlisting}
\end{frame}

\begin{frame}[fragile]
\begin{lstlisting}
let line ops a b x =
  let (+) = ops.plus
  let (*) = ops.times
  a * x + b
\end{lstlisting}
\end{frame}

\begin{frame}[fragile]
\begin{lstlisting}
let intOps   : NumberOperations<int>   = { plus = (+); times = (*) }
let floatOps : NumberOperations<float> = { plus = (+); times = (*) }
\end{lstlisting}
\end{frame}

\begin{frame}[fragile]
\begin{lstlisting}
let a = line intOps 1 2 3
let b = line floatOps 1.0 2.0 3.0
\end{lstlisting}
\end{frame}

\SlideSubSection{Creating a library of abstract operations}
\begin{slide}{
\item \texttt{plus} and \texttt{times} really look the same
\item We could build a record of operations for both of them
\item \texttt{Number} is then two of these records of operations
}\end{slide}

\begin{frame}[fragile]
\begin{lstlisting}
type Combine<'a> = { Empty : 'a; Append : 'a -> 'a -> 'a }
  with static member Create z (+) = { Empty = z; Append = (+) }
\end{lstlisting}
\end{frame}

\begin{frame}[fragile]
\begin{lstlisting}
let intPlus    = Combine<int>.Create 0 (+)
let intTimes   = Combine<int>.Create 1 (*)
let floatPlus  = Combine<float>.Create 0.0 (+)
let floatTimes = Combine<float>.Create 1.0 (*)
let stringPlus = Combine<string>.Create "" (+)
let listPlus() = Combine<List<'a>>.Create [] (@)
let setPlus()  = Combine<Set<'a>>.Create Set.empty (+)
\end{lstlisting}
\end{frame}

\begin{frame}[fragile]
\begin{lstlisting}
let sumStuff ops a b c =
  let (+) = ops.Append
  a + b + c + c
\end{lstlisting}
\end{frame}

\begin{frame}[fragile]
\begin{lstlisting}
sumStuff intPlus 10 20 30
\end{lstlisting}
\end{frame}

\begin{frame}[fragile]
\begin{lstlisting}
sumStuff floatPlus 10.0 20.0 30.0
\end{lstlisting}
\end{frame}

\begin{frame}[fragile]
\begin{lstlisting}
sumStuff stringPlus "a" "b" "c"
\end{lstlisting}
\end{frame}

\begin{frame}[fragile]
\begin{lstlisting}
sumStuff (listPlus()) [1] [2] [3;4]
\end{lstlisting}
\end{frame}

\begin{frame}[fragile]
\begin{lstlisting}
sumStuff (listPlus()) ["1"] ["2"] ["3";"4"]
\end{lstlisting}
\end{frame}

\begin{frame}[fragile]
\begin{lstlisting}
type Number<'a> = 
  { Plus  : Combine<'a>
    Times : Combine<'a> }
  with static member Create p t = { Plus = p; Times = t }
\end{lstlisting}
\end{frame}

\begin{frame}[fragile]
\begin{lstlisting}
let line ops a b x =
  let (+) = ops.Plus.Append
  let (*) = ops.Times.Append
  a * x + b
\end{lstlisting}
\end{frame}

\begin{frame}[fragile]
\begin{lstlisting}
let intOps     = Number<int>.Create intPlus intTimes
let floatOps   = Number<float>.Create floatPlus floatTimes
\end{lstlisting}
\end{frame}

\begin{frame}[fragile]
\begin{lstlisting}
let a = line intOps 1 2 3
let b = line floatOps 1.0 2.0 3.0
\end{lstlisting}
\end{frame}

\SlideSubSection{Creating an automated hierarchy of abstract operations}
\begin{slide}{
\item We can build generic combinators that transform our records of functions
\item This allows us to \textbf{\textit{automatically}} extend our library
}\end{slide}

\begin{frame}[fragile]
\begin{lstlisting}
let optionCombine (c:Combine<'a>) : Combine<Option<'a>> = 
  Combine<Option<'a>>.Create (Some c.Empty) (fun (x) (y) -> match x, y with Some x, Some y -> Some(c.Append x y) | _ -> None)
\end{lstlisting}
\end{frame}

\begin{frame}[fragile]
\begin{lstlisting}
let optionNumber (n:Number<'a>) : Number<Option<'a>> = 
  Number<Option<'a>>.Create (optionCombine n.Plus) (optionCombine n.Times)
\end{lstlisting}
\end{frame}

\begin{frame}[fragile]
\begin{lstlisting}
line (intOps |> optionNumber) (Some 3) (Some 10) (Some 5)
\end{lstlisting}
\end{frame}

\begin{frame}[fragile]
\begin{lstlisting}
let pairCombine (c1:Combine<'a>) (c2:Combine<'b>) : Combine<'a * 'b> = 
  Combine<'a * 'b>.Create (c1.Empty, c2.Empty) (fun (x1,y1) (x2,y2) -> c1.Append x1 x2, c2.Append y1 y2)
\end{lstlisting}
\end{frame}

\begin{frame}[fragile]
\begin{lstlisting}
let pairNumber (c1:Number<'a>) (c2:Number<'b>) : Number<'a * 'b> = 
  Number<'a * 'b>.Create (pairCombine c1.Plus c2.Plus) (pairCombine c1.Times c2.Times)
\end{lstlisting}
\end{frame}

\begin{frame}[fragile]
\begin{lstlisting}
line (pairNumber intOps floatOps) (3, 1.0) (4, 2.0) (5, 6.0)
\end{lstlisting}
\end{frame}

\begin{frame}[fragile]
\begin{lstlisting}
line (pairNumber intOps floatOps |> optionNumber) (Some(3, 1.0)) (Some(4, 2.0)) (Some(5, 6.0))
\end{lstlisting}
\end{frame}

\SlideSubSection{Conclusions and assignment}
\begin{slide}{
\item The assignments are on Natschool
\item \textbf{Restore} the games to a working state
\item Hand-in a \textbf{printed} report that only contains your \textbf{sources} and the associated \textbf{documentation}
}\end{slide}

\begin{slide}{
\item Any book on the topic will do
\item I did write my own (Friendly F\#) that I will be loosely following for the course, \textbf{but it is absolutely not mandatory or necessary to pass the course}
}\end{slide}


\begin{frame}{Dit is het}
\center
\fontsize{18pt}{7.2}\selectfont
The best of luck, and thanks for the attention!
\end{frame}

\end{document}
