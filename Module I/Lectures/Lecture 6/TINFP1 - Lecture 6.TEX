\documentclass{beamer}
\usetheme{Warsaw}
\usepackage{hrslides}
\usepackage{graphicx}
\usepackage[space]{grffile}
\usepackage{listings}
\lstset{language=C,
basicstyle=\ttfamily\footnotesize,
frame=shadowbox,
mathescape=true,
showstringspaces=false,
showspaces=false,
breaklines=true}
\usepackage[utf8]{inputenc}

\title{Accumulators, tail recursion, and continuation passing style}

\author{Dr. Giuseppe Maggiore}

\institute{Hogeschool Rotterdam \\ 
Rotterdam, Netherlands}

\date{}

\begin{document}
\maketitle

\SlideSection{Introduction}
\SlideSubSection{Lecture topics}
\begin{slide}{
\item Recursion uses the stack: we want to avoid stack overflow for deeply recursive problems
\item Recursion lends itself well to backtracking: we will cache recurring computations
\item Tail recursion can be automated with higher-order functions
}\end{slide}

\SlideSubSection{Accumulators and tail recursion}
\begin{slide}{
\item Recursion often uses the stack to store intermediate values
\item Lots of intermediate values are then unwinded while popping the stack
\item We can condense these values into an accumulator
\item Compiler removes stack usage for such functions
\item No stack overflow possible
}\end{slide}

\begin{frame}[fragile]
\begin{lstlisting}
let rec fact n = 
  match n with
  | 0 -> 1
  | _ -> n * fact (n-1)
\end{lstlisting}
\end{frame}

\begin{frame}[fragile]
\begin{lstlisting}
let factAcc n = 
  let rec aux n acc = 
    match n with
    | 0 -> acc
    | _ -> aux (n-1) (n * acc)
  aux n 1
\end{lstlisting}
\end{frame}

\begin{frame}[fragile]
\begin{lstlisting}
let sum l = 
  let rec aux l acc = 
    match l with
    | [] -> acc
    | x :: xs -> aux xs (x + acc)
  aux l 0
\end{lstlisting}
\end{frame}

\begin{frame}[fragile]
\begin{lstlisting}
let mul l = 
  let rec aux l acc = 
    match l with
    | [] -> acc
    | x :: xs -> aux xs (x * acc)
  aux l 1
\end{lstlisting}
\end{frame}

\begin{frame}[fragile]
\begin{lstlisting}
let reverse l = 
  let rec aux l acc = 
    match l with
    | [] -> acc
    | x :: xs -> aux xs (x :: acc)
  aux l []
\end{lstlisting}
\end{frame}

\begin{frame}[fragile]
\begin{lstlisting}
let foldAcc z f l = 
  let rec aux l acc = 
    match l with
    | [] -> acc
    | x :: xs -> aux xs (f x acc)
  aux l z
\end{lstlisting}
\end{frame}

\begin{frame}[fragile]
\begin{lstlisting}
let sum = foldAcc 0 (+)
let mul = foldAcc 1 (*)
let reverse = foldAcc [] (fun x xs -> x :: xs)
\end{lstlisting}
\end{frame}

\begin{frame}[fragile]
\begin{lstlisting}
let rec fibo n =
  match n with
  | 0 -> 0
  | 1 -> 1
  | _ -> fibo (n-1) + fibo (n-2)
\end{lstlisting}
\end{frame}

\begin{frame}[fragile]
\begin{lstlisting}
let fiboAcc n =
  let rec aux n curr prev =
    match n with
    | 0 -> 0
    | 1 -> curr
    | _ -> 
      aux (n-1) (curr+prev) curr
  aux n 1 0
\end{lstlisting}
\end{frame}

\SlideSubSection{Continuation passing style}
\begin{slide}{
\item We can automate the process of tail recursion
\item We define our functions so that they take \texttt{k}, an additional parameter
\item Instead of returning a result \texttt{x}, \textit{we give it to} \texttt{k} and return \texttt{k x}
\item We may create a new \texttt{k} as an anonymous function within which recursive calls happen
}\end{slide}

\begin{frame}[fragile]
\begin{lstlisting}
let rec factorialCPS n k = 
  if n <= 1 then k n
  else 
    factorialCPS (n-1) (fun fac_n_min_1 -> k (n * fac_n_min_1))
\end{lstlisting}
\end{frame}

\begin{frame}[fragile]
\begin{lstlisting}
let rec fibonacciCPS n k = 
  if n <= 1 then k n
  else 
    fibonacciCPS (n-1) (fun fib_n_min_1 -> 
                            fibonacciCPS (n-2) (fun fib_n_min_2 -> 
                                                    k (fib_n_min_1 + fib_n_min_2)))
\end{lstlisting}
\end{frame}

\begin{frame}[fragile]
\begin{lstlisting}
type Continuation<'a,'b> = ('a -> 'b) -> 'b

let (>>=) (k1 : Continuation<'a,'b>) (k2 : 'a -> Continuation<'a1,'b>) : Continuation<'a1,'b> = 
    fun (a1_to_b) -> k1 (fun a -> k2 a a1_to_b)

let (!) (x:'a) : Continuation<'a,'b> = fun (k:'a -> 'b) -> k x
\end{lstlisting}
\end{frame}

\begin{frame}[fragile]
\begin{lstlisting}
let rec factCPSMonadic n =
  if n <= 1 then !n
  else
    factCPSMonadic (n-1) >>= (fun n1 -> !(n * n1))
\end{lstlisting}
\end{frame}

\begin{frame}[fragile]
\begin{lstlisting}
let rec fibCPSMonadic n =
  if n <= 1 then !n
  else
    fibCPSMonadic (n-1) >>= (fun n1 -> 
    fibCPSMonadic (n-2) >>= (fun n2 ->
    !(n1+n2)))
\end{lstlisting}
\end{frame}

\SlideSubSection{Memoization}
\begin{slide}{
\item Inputs that are recurring have the same output
\item Same input yields same output only with referential transparency
\item We cache the results to avoid wasteful computation
}\end{slide}

\begin{frame}[fragile]
\begin{lstlisting}
let fiboMem n =
  let rec aux n mem =
    match mem |> Map.tryFind n with
    | Some res -> res,mem
    | None ->
      match n with
      | 0 -> 0,mem
      | 1 -> 1,mem
      | _ -> 
        let res1,mem'  = aux (n-1) mem
        let res2,mem'' = aux (n-2) (mem' |> Map.add (n-1) res1)
        res1 + res2,(mem'' |> Map.add (n-2) res2)
  aux n Map.empty |> fst
\end{lstlisting}
\end{frame}

\begin{frame}[fragile]
\begin{lstlisting}
let store (cache:Ref<Map<'a,'b>>) k input output =
  cache := cache.Value |> Map.add input output
  k output
\end{lstlisting}
\end{frame}

\begin{frame}[fragile]
\begin{lstlisting}
let memoize (cache:Ref<Map<'a,'b>>) f input k =
  match cache.Value |> Map.tryFind input with
  | Some output -> k output
  | None ->
    f cache input k
\end{lstlisting}
\end{frame}

\begin{frame}[fragile]
\begin{lstlisting}
let rec fibonacciCPSMem (cache:Ref<Map<int,int>>) n k = 
  if n <= 1 then store cache n n k
  else 
    memoize cache fibonacciCPSMem (n-1) 
      (fun fib_n_min_1 -> 
           memoize cache fibonacciCPSMem (n-2) 
             (fun fib_n_min_2 -> 
                   store cache n (fib_n_min_1 + fib_n_min_2) k))
\end{lstlisting}
\end{frame}


\SlideSubSection{Course conclusions}
\begin{slide}{
\item In the next course we will continue on building abstractions like \texttt{(>>=)} and \texttt{(!)} or \texttt{store} and \texttt{memoize}
\item We will also show how to combine such abstractions together
\item While building a small compiler for F\# itself (a small subset)
}\end{slide}


\SlideSubSection{Conclusions and assignment}
\begin{slide}{
\item The assignments are on Natschool
\item \textbf{Restore} the games to a working state
\item Hand-in a \textbf{printed} report that only contains your \textbf{sources} and the associated \textbf{documentation}
}\end{slide}

\begin{slide}{
\item Any book on the topic will do
\item I did write my own (Friendly F\#) that I will be loosely following for the course, \textbf{but it is absolutely not mandatory or necessary to pass the course}
}\end{slide}


\begin{frame}{Dit is het}
\center
\fontsize{18pt}{7.2}\selectfont
The best of luck, and thanks for the attention!
\end{frame}

\end{document}
