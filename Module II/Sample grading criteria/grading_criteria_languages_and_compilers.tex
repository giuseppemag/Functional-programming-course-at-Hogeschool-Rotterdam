\documentclass[10pt,a4paper]{article}
\usepackage[utf8]{inputenc}
\usepackage{csquotes}
\usepackage{fullpage}
\usepackage[english]{babel}
\usepackage{amsmath}
\usepackage{amsthm}
\usepackage{amsfonts}
\usepackage{amssymb}
\usepackage{graphicx}
\usepackage{listings}
\usepackage{paralist}
\usepackage{tikzsymbols}
\lstset{frame=lrbt}
\lstset{basicstyle=\footnotesize}

\lstset
{
	basicstyle=\scriptsize\ttfamily,
	breaklines=true,
	frame=single,
	showstringspaces=false,
	tabsize=2
}

\author{Dr. Giuseppe Maggiore}
\title{Grading criteria for languages and compilers course}

\begin{document}

\maketitle

\section{Introduction}
In this document the 

\section{Grading criteria}
...

\subsection{General grading criteria}
Purity, correctness, compilation, runtime errors.

Idiomatic style such as seen in the lectures.

\subsection{Assignment 3}


Assignment 3: Complete the definition of the typeCheck function so that it checks the type relations of terms.

Note: for bonus points you can make this function correctly handle higher-order-functions and generic types.


\paragraph{Sample solution}

...

\begin{lstlisting}
let typeCheck (p:Term) : Option<Type> =
  let rec typeCheck (p:Term) (env:Map<string,Type>) : Option<Type> =
    match p with
    | Var s -> env |> Map.tryFind s
    | Term.Int _ -> Some Type.Int
    | Term.Bool _ -> Some Type.Bool
    | Application(f,a) ->
      match typeCheck f env,typeCheck a env with
      | Some(Fun(fI,fO)),Some(a) when a = fI -> // TODO: generic instantiation should happen here
        Some fO
      | _ -> failwithf "Invalid function invocation %A" p
    | FunctionDef(arg,body) ->
      match typeCheck body (env |> Map.add arg.Name arg.Type) with
      | Some bT -> Some(Type.Fun(arg.Type, bT))
      | _ -> failwithf "Invalid function definition %A" p
    | Plus(a,b) ->
      match typeCheck a env,typeCheck b env with
      | Some(Int),Some(Int) -> Some Int
      | _ -> failwithf "Invalid sum %A" p
    | GreaterThan(a,b) ->
      match typeCheck a env,typeCheck b env with
      | Some(Int),Some(Int) -> Some Bool
      | _ -> failwithf "Invalid sum %A" p
  typeCheck p Map.empty
\end{lstlisting}


\paragraph{Specific criteria}
Answer model for this assignment.

This assignment requires building a function that exhibits the following characteristics in order to pass the assignment:
- the function is recursive
- the function is referentially transparent
- the function should handle types of functions such as int -> int
- the function should handle primitive types such as int or bool if the language supports them
- the function should represent information about variable bindings in the form of some mapping container 
  - the input of this container are variable identifiers (int's or strings)
  - the output of this container are values or terms
  - in the lambda calculus values are terms (thus there is no separate definition)
- (advanced) the function should handle generic type variables and generic instantiation of terms
  - with the same environment as that for variables
- (advanced) Option<'a> bindings are handled with a monad

The assignment is passed if at least five of the above-criteria are handled.

\end{document}